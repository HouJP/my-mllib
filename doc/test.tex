\documentclass[a4paper,11pt,         % A4纸
               % twoside,              % 双面
%              openany               % 新章节在偶数页开始
               ]{article}

\usepackage{xeCJK}

%%%%%%%%%% 版面控制 %%%%%%%%%%
\usepackage{indentfirst}             % 首行缩进
\iffalse
\usepackage[%paperwidth=18.4cm, paperheight= 26cm,
            body={14.6true cm,22true cm},
            twosideshift=0 pt,
            %headheight=1.0true cm
            ]{geometry}
\fi
\usepackage[perpage,symbol]{footmisc}% 脚注控制
\usepackage[sf]{titlesec}            % 控制标题
\usepackage{titletoc}                % 控制目录
\usepackage{fancyhdr}                % 页眉页脚
\usepackage{type1cm}                 % 控制字体大小
\usepackage{indentfirst}             % 首行缩进
\usepackage{makeidx}                 % 建立索引
\usepackage{textcomp}                % 千分号等特殊符号
\usepackage{layouts}                 % 打印当前页面格式
\usepackage{bbding}                  % 一些特殊符号
\usepackage{cite}                    % 支持引用
\usepackage{color,xcolor}            % 支持彩色文本、底色、文本框等
\usepackage{listings}                % 粘贴源代码
\lstloadlanguages{}                  % 所要粘贴代码的编程语言
\lstset{language=,tabsize=4, keepspaces=true,
    xleftmargin=2em,xrightmargin=2em, aboveskip=1em,
    backgroundcolor=\color{lightgray},    % 定义背景颜色
    frame=none,                      % 表示不要边框
    keywordstyle=\color{blue}\bfseries,
    breakindent=22pt,
    numbers=left,stepnumber=1,numberstyle=\tiny,
    basicstyle=\footnotesize,
    showspaces=false,
    flexiblecolumns=true,
    breaklines=true, breakautoindent=true,breakindent=4em,
    escapeinside={/*@}{@*/}
}

%%%%%%%%%% 字体支持 %%%%%%%%%%%%
%\usepackage{ccmap}                  % 使pdfLatex生成的文件支持复制等
\usepackage{CJK,CJKnumb,CJKulem}     % 中文支持
\usepackage{times}     % 包括 Times Roman + Helvetica + Courier
%\usepackage{palatino} % 包括 Palatino + Helvetica + Courier
%\usepackage{newcent}  % 包括 New Century Schoolbook + Avant Garde + Courier
%\usepackage{bookman}  % 包括 Bookman + Avant Garde + Courier

%%%%%%%%%% 代码 %%%%%%%%%%
% \usepackage[ruled, vlined, linesnumbered]{algorithm2e}
\usepackage{algorithm}
\usepackage{algorithmicx}
\usepackage{algpseudocode}

%%%%%%%%%% 数学符号公式 %%%%%%%%%%
\usepackage{latexsym}
\usepackage{amsmath}                 % AMS LaTeX宏包
\usepackage{amssymb}                 % 用来排版漂亮的数学公式
\usepackage{amsbsy}
\usepackage{amsthm}
\usepackage{amsfonts}
\usepackage{mathrsfs}                % 英文花体字体
\usepackage{bm}                      % 数学公式中的黑斜体
\usepackage{relsize}                 % 调整公式字体大小:\mathsmaller, \mathlarger
\usepackage{caption2}                % 浮动图形和表格标题样式

%%%%%%%%%% 图形支持宏包 %%%%%%%%%%
\usepackage{graphicx}
% \ifx\pdfoutput\undefined             % 用latex或pdflatex编译
%   \usepackage[dvips]{graphicx}       % 将eps格式的图片放在figures目录下
% \else                                % 在setup/format.tex中用以下命令注明路径:
%   \usepackage[pdftex]{graphicx}      % \graphicspath{{figures/}}
% \fi
%\usepackage{subfigure}
\usepackage{epsfig}                  % 支持eps图像
%\usepackage{picinpar}               % 图表和文字混排宏包
%\usepackage[verbose]{wrapfig}       % 图表和文字混排宏包
%\usepackage{eso-pic}                % 向文档的部分页加n副图形, 可实现水印效果
%\usepackage{eepic}                  % 扩展的绘图支持
%\usepackage{curves}                 % 绘制复杂曲线
%\usepackage{texdraw}                % 增强的绘图工具
%\usepackage{treedoc}                % 树形图绘制
%\usepackage{pictex}                 % 可以画任意的图形\usepackage{amsbsy}
\usepackage{amsthm}
\usepackage{amsfonts}
\usepackage{mathrsfs}                % 英文花体字体
\usepackage{bm}                      % 数学公式中的黑斜体
\usepackage{relsize}                 % 调整公式字体大小:\mathsmaller, \mathlarger
\usepackage{caption2}                % 浮动图形和表格标题样式

%%%%%%%%%% 表格支持宏包 %%%%%%%%%%
\usepackage{multirow}

%%%%%%%%%% 图形支持宏包 %%%%%%%%%%
\usepackage{graphicx}
% \ifx\pdfoutput\undefined             % 用latex或pdflatex编译
%   \usepackage[dvips]{graphicx}       % 将eps格式的图片放在figures目录下
% \else                                % 在setup/format.tex中用以下命令注明路径:
%   \usepackage[pdftex]{graphicx}      % \graphicspath{{figures/}}
% \fi
%\usepackage{subfigure}
\usepackage{epsfig}                  % 支持eps图像
%\usepackage{picinpar}               % 图表和文字混排宏包
%\usepackage[verbose]{wrapfig}       % 图表和文字混排宏包
%\usepackage{eso-pic}                % 向文档的部分页加n副图形, 可实现水印效果
%\usepackage{eepic}                  % 扩展的绘图支持
%\usepackage{curves}                 % 绘制复杂曲线
%\usepackage{texdraw}                % 增强的绘图工具
%\usepackage{treedoc}                % 树形图绘制
%\usepackage{pictex}                 % 可以画任意的图形
%\usepackage{hyperref}

%%%%%%%%%% 一些距离设置 %%%%%%%%%%%
\setlength{\floatsep}{10pt plus 3pt minus 2pt}       % 图形之间或图形与正文之间的距离
\setlength{\abovecaptionskip}{2pt plus 1pt minus 1pt}% 图形中的图与标题之间的距离
\setlength{\belowcaptionskip}{3pt plus 1pt minus 2pt}% 表格中的表与标题之间的距

%%%%%%%%%% 纸张和页面的大小 %%%%%%%%%%
%\paperwidth   20 true cm            % 纸张宽
%\paperheight  30 true cm            % 纸张高
%\textwidth    10 true cm            % 正文宽
%\textheight   20 true cm            % 正文高
%\headheight      14pt               % 页眉高
%\headsep         16pt               % 页眉距离
%\footskip        27pt               % 页脚距离
%\marginparsep    10pt               % 边注区距离
%\marginparwidth  100pt              % 边注区宽
\makeindex                           % 生成索引
% \pagestyle{fancy}                    % 页眉页脚风格
\fancyhf{}                           % 清空当前页眉页脚的默认设置

%%%%%%%%%% 伪代码 %%%%%%%%%%%
% \floatname{algorithm}{算法}
% \renewcommand{\algorithmicrequire}{\textbf{输入:}}
% \renewcommand{\algorithmicensure}{\textbf{输出:}}

%%%%%%%%%% 导入中文环境 %%%%%%%%%%
% \AtBeginDocument{\begin{CJK*}{GBK}{song} % 不计中文的空格
% \CJKindent                           % 首行缩进两个汉字
% \sloppy\CJKspace                     % 中英文混排的断行
% \CJKtilde                            % 重新定义~,用~隔开中英文
% \CJKcaption{GB}                      % 章节标题的中文化
% }
% \AtEndDocument{\end{CJK*}}

%%%%%%%%%% 正文 %%%%%%%%%%
\begin{document}

%%%%%%%%%% 一些新定义 %%%%%%%%%%
\newcommand{\song}{\CJKfamily{song}} % 宋体
\newcommand{\hei}{\CJKfamily{hei}}   % 黑体
\newcommand{\fs}{\CJKfamily{fs}}     % 仿宋
\newcommand{\kai}{\CJKfamily{kai}}   % 楷体

%%%%%%%%%% 定理类环境的定义 %%%%%%%%%%
%% 必须在导入中文环境之后
\newtheorem{example}{例}             % 整体编号
% \newtheorem{algorithm}{算法}
\newtheorem{theorem}{定理}[section]  % 按 section 编号
\newtheorem{definition}{定义}
\newtheorem{axiom}{公理}
\newtheorem{property}{性质}
\newtheorem{proposition}{命题}
\newtheorem{lemma}{引理}
\newtheorem{corollary}{推论}
\newtheorem{remark}{注解}
\newtheorem{condition}{条件}
\newtheorem{conclusion}{结论}
\newtheorem{assumption}{假设}

%%%%%%%%%% 一些重定义 %%%%%%%%%%
%% 必须在导入中文环境之后
\renewcommand{\contentsname}{目录}     % 将Contents改为目录
\renewcommand{\abstractname}{摘\ \ 要} % 将Abstract改为摘要
\renewcommand{\refname}{参考文献}      % 将References改为参考文献
\renewcommand{\indexname}{索引}
\renewcommand{\figurename}{图}
\renewcommand{\tablename}{表}
\renewcommand{\appendixname}{附录}
\renewcommand{\proofname}{\hei 证明}
% \renewcommand{\algorithm}{\hei 算法}

%%%%%%%%%% 重定义字号命令 %%%%%%%%%%
\newcommand{\yihao}{\fontsize{26pt}{36pt}\selectfont}       % 一号, 1.4倍行距
\newcommand{\erhao}{\fontsize{22pt}{28pt}\selectfont}       % 二号, 1.25倍行距
\newcommand{\xiaoer}{\fontsize{18pt}{18pt}\selectfont}      % 小二, 单倍行距
\newcommand{\sanhao}{\fontsize{16pt}{24pt}\selectfont}      % 三号, 1.5倍行距
\newcommand{\xiaosan}{\fontsize{15pt}{22pt}\selectfont}     % 小三, 1.5倍行距
\newcommand{\sihao}{\fontsize{14pt}{21pt}\selectfont}       % 四号, 1.5倍行距
\newcommand{\bansi}{\fontsize{13pt}{19.5pt}\selectfont}     % 半四, 1.5倍行距
\newcommand{\xiaosi}{\fontsize{12pt}{18pt}\selectfont}      % 小四, 1.5倍行距
\newcommand{\dawu}{\fontsize{11pt}{11pt}\selectfont}        % 大五, 单倍行距
\newcommand{\wuhao}{\fontsize{10.5pt}{10.5pt}\selectfont}   % 五号, 单倍行距

%%%%%%%%%% 表格 %%%%%%%%%%
\newcommand{\tabincell}[2]{\begin{tabular}{@{}#1@{}}#2\end{tabular}}

%%%%%%%%%% 页眉和页脚的设置 %%%%%%%%%%
\lhead{一个~\LaTeX+CJK~的简单模板}
\rhead{\TeX~爱好者}
\lfoot{用~\LaTeX~写科技论文}
\rfoot{~\thepage~}

%%%%%%%%%% 论文标题、作者等 %%%%%%%%%%
\title{GBDT算法}
\author{BDA小组\\                     % 作者
        中国科学院计算技术研究所}
\date{2015年10月}                % 日期
\maketitle                           % 生成标题
% \tableofcontents                     % 插入目录
% \thispagestyle{empty}                % 首页无页眉页脚

%%%%%%%%%% section %%%%%%%%%%
\section{GBDT基础}

\subsection{算法概述}

GBDT(Gradient Boosting Decision Tree)是一种迭代的回归决策树算法。该算法生成的模型由多棵决策回归树构成,所有树的预测值累加起来即为最终的预测值。GBDT是一种泛化能力(Generalization)较强的算法,目前被广泛应用与搜索排序,并经常出现在机器学习领域的竞赛中。\\

GBDT由两个重要的概念组成:回归决策树(Regression Decision Trees)和梯度迭代(Gradient Boosting)。

\subsection{回归决策树}

算法实现的第一个版本中,暂不支持离散特征的输入,因此这里假设输入特征均为连续值。\\

回归决策树是GBDT模型的基本组成单位,用来做回归分析。回归决策树采用二分递归分隔的方法,在每个节点处,选取某个特征的某个阈值,将当前节点所包含的样本集分成两个子样本集,形成两个子节点,再对子节点采用同样的方法继续分割,直到不能分割为止。因此,每个回归决策树都是一棵结构简单的二叉树。

\subsubsection{树节点的分裂}

用回归树上每个节点的方差(Variance)来表示该节点的不纯度(Impurity),方差越大,不纯度越高:
\begin{eqnarray}
	\label{impurity1}
{I}_{v}(N)=\frac{1}{{\left|S \right|}^{2}}\sum_{i\in S}\sum_{j\in S}\frac{1}{2}(x_{i}-x_{j})^2
\end{eqnarray}
其中,S表示节点所包含的样本集。\\

为了提高算法效率,使用公式~(\ref{impurity1})~的另一种表示形式:
\begin{eqnarray}
	\label{impurity2}
{I}_{v}(N)=\frac{1}{\left | S \right |}\left ( \sum_{i\in S} {x_{i}}^2 - \frac{1}{\left | S \right |} \sum_{i \in S} {x_{i}}^2 \right )
\end{eqnarray}
\\

根据节点及其左右孩子节点的方差,用下式可以得到每次分裂的方差缩减量(Variance Reduction),方差缩减量越大,说明这次分裂效果越好:
\begin{eqnarray}
R_{v}(N) = I_{v}(N) - \left ( \frac{\left | S_{1}\right |}{\left | S \right |} I_{v}\left ( N_{1} \right ) + \frac{\left | S_{2}\right |}{\left | S \right |} I_{v}\left ( N_{2} \right )\right)
\end{eqnarray}
其中,~$N$~表示父亲节点所包含的样本集,~$N_{1}$~表示左孩子节点所包含的样本集,~$N_{2}$~表示右孩子节点所包含的样本集。
\\

这样,在每个节点处,先分别对所有特征进行排序,并依次取每维特征的特征值作为阈值进行分裂尝试,找到可以使得方差缩减量最大的特征值~$f_{ij}$~,即为该节点的分裂标准。 

\subsubsection{树节点的预测值}

当节点不能再继续分裂的时候,我们使用该节点上样本集的均值作为该节点所对应的预测值:
\begin{eqnarray}
P_{i} = \frac{1}{\left | S_{i} \right |} \sum_{i \in S_{i}} x_{i}
\end{eqnarray}
其中,~$P_{i}$~表示第i个叶子节点的预测值,~$S_{i}$~表示第i个叶子节点上的样本集。

\subsection{梯度迭代}
梯度迭代(Grandient Boosting)告诉我们如何将回归决策树进行组合。 \\

初始时,所有样本的预测值置为统一的值,该值使得损失函数的取值最小。在每一轮迭代的过程中,每个样本点使用损失函数的负梯度作为目标值构建回归树,更新预测值为本轮回归树的输出值及之前回归树的输出值之和。 \\

梯度迭代的执行过程请参考伪代码~(\ref{alg:gb})~。

\begin{algorithm}
	\caption{Gradient tree boosting for multiple additive regression trees}
	\label{alg:gb}
	\begin{algorithmic}[1] %每行显示行号
	\State $f_{0} \gets argmin_{\gamma} \sum_{i = 1}^{N} L \left ( y_{i}, \gamma\right )$
	\For{$m = 1 \to M$}
		\For{$i = 1 \to N$}
			\State $r_{im} \gets -\left [ \frac{\partial L\left ( y_{i}, f\left ( x_{i} \right ) \right )}{\partial f\left ( x_{i} \right )} \right ]_{f=f_{m-1}}$			
		\EndFor
		\State Fit a regression tree to the targets $r_{im}$ giving terminal regions $R_{jm}, j = 1,2,...,J_{m}$.
		\For{$j = 1 \to J_{m}$}
			\State $\gamma_{jm} \gets argmin_{\gamma} \sum_{x_{i} \in R_{jm}} L\left ( y_{i},f_{m-1}\left ( x_{i} \right ) + \gamma \right )$
		\EndFor
		\State $f_{m}\left ( x \right ) \gets f_{m-1}\left ( x \right ) + \sum_{j=1}^{J_{m}} \gamma_{jm}I\left ( x \in R_{jm} \right )$
		\label{alg:gb:update}
	\EndFor
	\State Output $\hat{f}\left ( x \right ) \gets f_{M}\left ( x \right )$
	\end{algorithmic}
\end{algorithm}

\subsection{缩减}

缩减(Shrinkage)的思想认为,在模型的迭代过程中,每次走一小步逐渐逼近结果的效果,要比每次都一大步很快逼近结果的方式更容易避免过拟合,同时也更容易收敛。 \\

也就是说,我们不能完全信任每一棵决策树,每一棵决策树只能学到真理的一小部分,累加的时候只累加一小部分,通过多棵决策树来弥补不足。 \\

因此,对伪代码~($\ref{alg:gb}$)~第~($\ref{alg:gb:update}$)~行的公式进行修改,如公式~($\ref{eqn:learning_rate}$)~所示。其中,$\rho_{m}$通常被称作学习率(Learning Rate)。
\begin{eqnarray}
	\label{eqn:learning_rate}
	f_{m}\left ( x \right ) \gets f_{m-1}\left ( x \right ) + \rho_{m} \sum_{j=1}^{J_{m}} \gamma_{jm}I\left ( x \in R_{jm} \right )
\end{eqnarray}


\section{GBDT实现}

目前GBDT算法的实现包括单机和半分布式两个版本。

\subsection{单机版本}

\subsubsection{回归决策树}

单机版本的回归决策树主要包括以下几个类:
\begin{itemize}
	\item bda.local.ml.DTree:由用户指定回归决策树参数,通过fit方法读取训练数据并输出回归树模型。
	\item bda.local.ml.model.DTreeModel:回归决策树模型类。该类包括训练参数及生成的二叉树数据结构。
	\item bda.local.ml.model.Node:回归决策树的节点类。该类记录了在对应节点处分裂所使用的特征编号及其分裂阈值。
	\item bda.local.ml.model.Stat:模型训练的过程中,节点的状态类。该类记录了对应节点所包含的训练数据范围及训练数据的个数、加和以及平方和,通过这些信息可以有效地降低模型训练的时间复杂度。
	\item bda.local.para.DTreePara:回归决策树参数类。它是决策树模型训练的依据。
\end{itemize}

DTree在训练模型的时候,会从根节点开始,通过遍历所有特征及其所有可能的分裂点的形式,为每一个节点寻找一个最佳的分裂特征及分裂阈值,并进行分裂。这个过程会不断进行下去,直到所有的节点都不满足分裂条件为止。

\subsubsection{梯度迭代}

单机版本的梯度迭代由以下几个类组成:
\begin{itemize}
	\item bda.local.ml.GBoost:由用户指定回归决策树及梯度迭代的参数,通过fit方法读取训练数据并输出GBDT模型。
	\item bda.local.ml.model.GBoostModel:GBDT模型类。该类包括训练参数及生成的GBDT模型(多个DTreeModel的集合)。
	\item bda.local.ml.para.GBoostPara:GBDT参数类。它是GBDT模型训练的依据。
\end{itemize}

\subsection{半分布式版本}

GBDT算法的分布式实现主要瓶颈在于回归决策树的分布式实现。 \\

根据理论部分的介绍,我们知道在构建一棵回归树的过程中需要遍历所有特征及其所有可能的分裂点。假设数据集的大小为$N$,而特征个数为$M$,那么在不计排序开销的前提下,回归树构建每一层的复杂度依然高达$O(N*M)$,这样的复杂度在大规模数据下的时间开销难以让人接受。 \\

为了减低算法的复杂度,使其可以满足大数据的需求,mllib采用的方式是在训练回归树模型之前,对数据进行采样,并对这些数据的特征值进行分箱操作,将分箱的阈值作为对应特征的备选分裂点。由于分箱的个数远远小于数据规模,这样,就可以将回归树构建每一层的复杂度由$O(N*M)$降为$O(N'*M)$,其中,$N'$表示分箱所使用的阈值个数。 \\

但是,GBoost在分布式的实现过程中并没有采用这一思想,而是在训练每一个回归树的模型过程时,采样训练样本,使用有限个数的样本训练单机版回归树模型作为单个弱分类器。这样,只对梯度迭代算法进行了分布式实现,从而形成了半分布式版本。

\subsubsection{梯度迭代}

分布式版本的梯度迭代由以下几个类组成:
\begin{itemize}
	\item bda.spark.ml.GBoost:由用户指定回归决策树及梯度迭代的参数,通过fit方法读取训练数据并输出GBDT模型。
	\item bda.spark.ml.model.GBoostModel:GBDT模型类。该类包括训练参数及生成的GBDT模型(多个DTreeModel的集合)。
	\item bda.spark.ml.para.GBoostPara:GBDT参数类。它是GBDT模型训练的依据。
\end{itemize}

\section{GBDT评测}

\subsection{单机版本}

在单机环境下,与时下流行的GBDT多线程开源实现xgboost进行了对比,如表~(\ref{tb:stand_along})~所示。

\begin{itemize}
	\item 测试环境:单机
		\begin{itemize}
			\item CPU: 1.3 GHz Intel Core i5
			\item Memory: 4 GB 1600 MHz DDR3
		\end{itemize}
	\item 数据集:cadata
		\begin{itemize}
			\item 类型:回归数据集
			\item 来源:http://lib.stat.cmu.edu/datasets/houses.zip
			\item 大小:6.4M
			\item \# of data:20,640
			\item \# of feature:8
		\end{itemize}
	\item 固定参数
		\begin{itemize}
			\item xgboost:gamma(0), max\_delata\_step(0), subsample(1), colsample\_byree(1), lambda(1), alpha(0), nthread(10), learning\_rate(0.01)
			\item bda.local.ml.GBoost:learning\_rate(0.01)
		\end{itemize}
\end{itemize}

\begin{table}%[!h]     % 强制在原位显示表格
% \resizebox{\textwidth}{!}{
\centering
\caption{单机版本评测}
\label{tb:stand_along}
\begin{tabular}{l|c|c|c|c|c|c}
% \hline
\hline
algorithm & \tabincell{c}{num\\ iter} & \tabincell{c}{max\\ depth} & \tabincell{c}{min\\ child\\ weight\\} & \tabincell{c}{total\\ time} & \tabincell{c}{train\\ RMSE} & \tabincell{c}{test\\ RMSE} \\
\hline
\multirow{5}{*}{xgboost} 	&	100		&	15	&	10	&	4s	&	97536.96	&	102027.12	\\
							&	300		&	15	&	10	&	9s	&	48573.12	&	58397.00	\\
							&	600		&	15	&	10	&	26s	&	21310.80	&	47144.61	\\
							&	600		&	15	&	20	&	24s	&	28383.57	&	47840.43	\\
							&	500		&	20	&	20	&	26s	&	28251.16	&	46656.16	\\
\hline
\hline
algorithm & \tabincell{c}{num\\ iter} & \tabincell{c}{max\\ depth} & \tabincell{c}{min\\ node\\ size\\} & \tabincell{c}{ave\\ iter\\ time} & \tabincell{c}{train\\ RMSE} & \tabincell{c}{test\\ RMSE} \\
\hline
\multirow{4}{*}{GBoost}		&	100		&	15	&	10	&	857ms	&	46993.31	&	60005.90	\\
							&	300		&	15	&	10	&	692ms	&	20494.19	&	47727.48 	\\
							&	600		&	15	&	10	&	896ms	&	15688.38	&	47232.18	\\
							&	1000	&	15	&	10	&	798ms	&	12343.36	&	47144.03	\\

\hline
% \hline

%   $_X$\hspace{3mm} $^Y$&$y_1$&$y_2$&$\cdots$&$y_j$&$\cdots$\\
% \hline
% $x_1$   &$p_{11}$&$p_{12}$&$\cdots$&$p_{1j}$&$\cdots$&$p_{1\cdot}$\\
% $x_2$   &$p_{21}$&$p_{22}$&$\cdots$&$p_{2j}$&$\cdots$&$p_{2\cdot}$\\
% $\vdots$&$\vdots$&$\vdots$&$\vdots$&$\vdots$&$\vdots$&$\vdots$ \\
% $x_i$   &$p_{i1}$&$p_{i2}$&$\cdots$&$p_{ij}$&$\cdots$&$p_{i\cdot}$\\
% $\vdots$&$\vdots$&$\vdots$&$\vdots$&$\vdots$&$\vdots$&$\vdots$ \\
% \hline
%    &$p_{\cdot 1}$&$p_{\cdot 2}$&$\cdots$&$p_{\cdot j}$&$\cdots$&1
% \label{marginal distribution}
\end{tabular}
\end{table}

\subsection{半分布式版本}
TODO


% \TeX~有诸如AMS\TeX、\LaTeX~等宏库。在~FreeBSD~下,缺省的宏库是~te\TeX。
% Knuth~用~\$~符号界定数学公式,意味着每个好的公式都是无价之宝。
% 有了~\TeX~系统,输入数学公式变得简单愉快。如,
% \begin{theorem}[L\'{e}vy\index{L\'{e}vy~定理}]
% 令~$F(x),\varphi(t)$~分别为随机变量~$X$~的分布函数和特征函数。
% 假定~$F(x)$~在~$a+h$~和~$a-h (h>0)$~处连续,则有
% \begin{eqnarray}
%   \label{Levy theorem}  % 方程的标记可以是专有名词
% F(a+h)-F(a-h)&=&\lim_{T\rightarrow\infty} \frac{1}{\pi}\int^{T}_{-T} \frac{\sin ht}{t} e^{-ita} \varphi(t)dt
% \end{eqnarray}
% \end{theorem}
% \begin{proof}
%   从略。感兴趣的读者可以参考……。
% \end{proof}
% L\'{e}vy~定理在分布函数和特征函数之间搭建了一座桥梁。由公式~(\ref{Levy theorem})~可得
% \begin{eqnarray}
%   \label{DensityCharacteristic}   % 自定义的标记
%   f(x)&=&\frac{1}{2\pi}\int^{+\infty}_{-\infty} e^{-itx}\varphi(t)dt
% \end{eqnarray}

% \begin{eqnarray}
%   \frac{F(x+\Delta x)-F(x)}{\Delta x}&=&\frac{1}{2\pi} \int^{+\infty}_{-\infty}
% \frac{\sin(t\Delta x/2)}{t\Delta x/2} e^{-it(x+\Delta x/2)} \varphi(t) dt\nonumber\\
%   f(x)&=&\frac{1}{2\pi} \int^{+\infty}_{-\infty} \lim_{\Delta x\rightarrow 0}
% \frac{\sin(t\Delta x/2)}{t\Delta x/2} e^{-it(x+\Delta x/2)} \varphi(t) dt\nonumber\\
%   &=&\frac{1}{2\pi}\int^{+\infty}_{-\infty} e^{-itx}\varphi(t)dt\nonumber
% \end{eqnarray}
% 我们知道特征函数的定义是
% \begin{eqnarray}
%   \label{section1:characteristic}   % 标记中注明了章节号
%   \varphi(t)&=& E(e^{itX})\nonumber\\
%             &=& \int^{+\infty}_{-\infty} e^{itx} f(x)dx
% \end{eqnarray}
% 对比~(\ref{DensityCharacteristic})~和~(\ref{section1:characteristic})~可见,
% 密度函数和特征函数之间的关系非常巧妙。
% % \end{proof}
% \HandRight 在~\TeX~环境里,数学公式的表达是很自然的,绝大多数命令就是英文的数学
% 专有名词或它们的缩写,如果你以前读过英文的数学文献,记忆这些命令是不难的。
% 手头有个命令快速寻查表是很方便的,
% 我用的是~Hypertext Help with \LaTeX,网上可以搜到,是免费的。

% %%%%%%%%%% section %%%%%%%%%%
% \section{符号、字体、颜色等}
% \begin{itemize}
%   \item 特殊字符:\# \$ \% \^{} \& \_ \{ \} \~{} $\backslash \cdots$
%   \item 中文字体:{\song 宋体} {\kai 楷体} {\hei 黑体} {\fs 仿宋}
%   \item 字体大小:{\tiny tiny} {\small small} {\normalsize normalsize}
%                   {\large large} {\Large Large} {\huge huge} {\Huge Huge}
%   \item 汉字大小:{\wuhao 五号} {\dawu 大五} {\xiaosi 小四} {\sihao 四号}
%                   {\xiaosan 小三} {\sanhao 三号} {\xiaoer 小二} {\erhao 二号} {\yihao 一号}
%   \item 各种颜色:{\color{red} 红色} {\color{yellow} 黄色} {\color{blue} 蓝色}
%                   {\color{magenta} 洋红} {\color{cyan} 蓝绿}
% \end{itemize}
% %%%%%%%%%%% section %%%%%%%%%%
% \section{图形表格等浮动对象}

% \index{贝叶斯方法}贝叶斯方法~\cite{Gelman}~主要用于小样本数据分析,它利用参数先验分布和
% 后验分布之差异进行统计推断,其一般步骤是:
% \begin{enumerate}
%   \item 构建概率模型,包括参数的先验分布。
%   \item 给定观察数据,计算参数的后验分布。
%   \item 分析模型的效果,如有必要,回到第一步。
% \end{enumerate}
% 下面,我们给一个表格的例子:
% \begin{center}
% \begin{table}[!h]     % 强制在原位显示表格
% \centering
% \caption{二维随机向量$(X,Y)$的边缘分布}
% \begin{tabular}{l|ccccc|c}
%   $_X$\hspace{3mm} $^Y$&$y_1$&$y_2$&$\cdots$&$y_j$&$\cdots$\\
% \hline
% $x_1$   &$p_{11}$&$p_{12}$&$\cdots$&$p_{1j}$&$\cdots$&$p_{1\cdot}$\\
% $x_2$   &$p_{21}$&$p_{22}$&$\cdots$&$p_{2j}$&$\cdots$&$p_{2\cdot}$\\
% $\vdots$&$\vdots$&$\vdots$&$\vdots$&$\vdots$&$\vdots$&$\vdots$ \\
% $x_i$   &$p_{i1}$&$p_{i2}$&$\cdots$&$p_{ij}$&$\cdots$&$p_{i\cdot}$\\
% $\vdots$&$\vdots$&$\vdots$&$\vdots$&$\vdots$&$\vdots$&$\vdots$ \\
% \hline
%    &$p_{\cdot 1}$&$p_{\cdot 2}$&$\cdots$&$p_{\cdot j}$&$\cdots$&1
% \label{marginal distribution}
% \end{tabular}
% \end{table}
% \end{center}
% 在表~\ref{marginal distribution}中,$p_{\cdot j}=\sum\limits_i p_{ij}$,类似地,$ p_{i\cdot}=\sum\limits_j p_{ij}$。
% % 插入一个图片
% %\includegraphics[width=50mm,height=40mm]{figures/demo.eps}

% %%%%%%%%%% section %%%%%%%%%%
% \section{生成索引}
% 键入命令:makeindex  文件名。\newline
% \indent 譬如对这个模板,生成~Template4CJK.ind~的过程如下。
% \begin{lstlisting}
% $ makeindex Template4CJK
% This is makeindex, version 2.14 [02-Oct-2002] (kpathsea + Thai support).
% Scanning input file Template4CJK.idx....done (4 entries accepted, 0 rejected).
% Sorting entries....done (9 comparisons).
% Generating output file Template4CJK.ind....done (18 lines written, 0 warnings).
% Output written in Template4CJK.ind.
% Transcript written in Template4CJK.ilg.
% \end{lstlisting}

% \printindex % 打印出索引名及其所在页码,即那些\index{索引名}
% %%%%%%%%%% 参考文献 %%%%%%%%%%
% \begin{thebibliography}{}
% \bibitem[Gelman et~al., 2004]{Gelman} Gelman, A., Carlin, J.~B., Stern, H.~S.  \& Rubin, D.~B. (2004)
% Bayesian Data Analysis (Second Edition).  \newblock Chapman \& Hall/CRC.
% \end{thebibliography}
\clearpage
\end{document}
%%%%%%%%%% 结束 %%%%%%%%%%

